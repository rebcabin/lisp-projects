% Created 2016-02-22 Mon 09:20
\documentclass[10pt,oneside,x11names]{article}
\usepackage[utf8]{inputenc}
\usepackage[T1]{fontenc}
\usepackage{fixltx2e}
\usepackage{graphicx}
\usepackage{grffile}
\usepackage{longtable}
\usepackage{wrapfig}
\usepackage{rotating}
\usepackage[normalem]{ulem}
\usepackage{amsmath}
\usepackage{textcomp}
\usepackage{amssymb}
\usepackage{capt-of}
\usepackage{hyperref}
\usepackage{geometry}
\usepackage{palatino}
\usepackage{siunitx}
\usepackage{braket}
\usepackage[euler-digits,euler-hat-accent]{eulervm}
\author{Brian Beckman}
\date{\textit{<2016-02-21 Sun>}}
\title{Storeys}
\hypersetup{
 pdfauthor={Brian Beckman},
 pdftitle={Storeys},
 pdfkeywords={},
 pdfsubject={},
 pdfcreator={Emacs 24.5.1 (Org mode 8.2.10)}, 
 pdflang={English}}
\begin{document}

\maketitle
\setcounter{tocdepth}{2}
\tableofcontents


\section{Prelude}
\label{sec:orgheadline1}

\begin{verbatim}
;; Copyright (c) 2016 Brian Beckman
\end{verbatim}

``Storeys'' is a pun on ``story'' and the levels of a dungeon. We avoid the word
``level'' because it's ambiguous between the level of advancement of a character
and the level or storey of the dungeon.

\section{Project Structure}
\label{sec:orgheadline4}

\subsection{Brute-Force Loads}
\label{sec:orgheadline2}

The current system structure is just a sequence of load commands. Dependencies
of one facility on another is expressed only implicitly by the sequence of those
loads, and only in top-level, runnable lisp files. For example, \emph{box} depends on
\emph{point}, but \texttt{box.lisp} doesn't load \texttt{point.lisp}. Instead, any file that wants
to use \emph{box} must load \texttt{point.lisp} before loading \texttt{box.lisp}. \texttt{box.lisp}
\emph{could} load \texttt{point.lisp}, but then \texttt{point.lisp} would be loaded more than once.
Harmless but inefficient. If \texttt{box.lisp} doesn't load \texttt{point.lisp}, then
\texttt{box.lisp} can't stand alone. That's the decision we made:

\textbf{\emph{Facilities are not typically standalone, and their dependencies are implicit in the sequence of load commands.}}

Currently, there are two contexts in which this brute-force loading goes on:
\begin{enumerate}
\item any main code, such as \texttt{charms-test-5.lisp}
\item quickcheck code, like \texttt{test.lisp}
\end{enumerate}

For example, in one intermediate version, \texttt{charms-test-5.lisp} loaded files in
this order:
\begin{verbatim}
(load "point.lisp")
(load "box.lisp")
(load "glyph.lisp")
(load "world.lisp")
(load "storey.lisp")
(load "room.lisp")
(load "rendering.lisp")
\end{verbatim}

The convention for quickcheck code is a little more subtle: it will load
unit-test files out of the \texttt{test} directory in the same order as a main file
does. For instance, the version of \texttt{test.lisp} corresponding to the version of
\texttt{charms-test-5.lisp} noted above looks like this:

\begin{verbatim}
(load "~/quicklisp/setup.lisp")
\end{verbatim}

\begin{verbatim}
(ql:quickload :cl-quickcheck)
(ql:quickload :defenum)
(ql:quickload :alexandria)
(ql:quickload :hash-set)
\end{verbatim}

\begin{verbatim}
(let ((*random-state* (make-random-state t))
      (*print-length* 6)
      (*load-verbose* t))

  (shadow 'cl-quickcheck:report '#:cl-user)
  (shadow 'defenum:enum         '#:cl-user)
  (use-package :cl-quickcheck)
  (use-package :defenum)

  ;; This sequence of 'load' expressions should parallel the 'load's at the head
  ;; of any main like 'charms-test-5.lisp'.

  (load "test/point.lisp")
  (load "test/box.lisp")
  (load "test/glyph.lisp")
  (load "test/geo.lisp")
  (load "test/world.lisp")
  (load "test/storey.lisp")
  (load "test/room.lisp")
  (load "test/rendering.lisp")
  )
\end{verbatim}

No doubt, you've deduced that main codes and test codes --- anything that runs
facilityy code --- must also load \emph{quickload} and any quickload dependencies
like \emph{alexandria}.  Just as with our own facilities, you must do this by brute
force. 

Each file in the test directory must load its brother from the top-level
directory. For example, \texttt{test/glyph.lisp} looks like this:

\begin{verbatim}
(load "glyph.lisp")

(quickcheck
 (is= 42 42))
\end{verbatim}

Because we run \texttt{test.lisp} from the top level, \texttt{test/glyph.lisp} will load the
top-level \texttt{glyph.lisp}. Of course, \texttt{test/glyph.lisp} must load any of its
dependencies. It doesn't have any in the version we're talking about, so
everything is fine. But \texttt{test/box.lisp} needs top-level \texttt{point.lisp}. So
\texttt{test/box.lisp} looks like this:

\begin{verbatim}
(load "point.lisp")
(load "box.lisp")

;;; blah blah blah

(quickcheck
 (is= 42 42))
\end{verbatim}

All this is cumbersome and brittle, so we will fix it with something like ASDF
later.

\subsection{{\bfseries\sffamily TODO} ASDF}
\label{sec:orgheadline3}

\section{Basics}
\label{sec:orgheadline8}
\subsection{Point}
\label{sec:orgheadline5}
\subsection{Box}
\label{sec:orgheadline6}
\subsection{Glyph}
\label{sec:orgheadline7}
\section{Storeys}
\label{sec:orgheadline55}

\subsection{World}
\label{sec:orgheadline9}

The world has a sequence of \emph{storeys}. There is a first one, but not a last one.

\begin{verbatim}
(defclass world ()
  ((storeys :accessor world-storeys :initform () :initarg :storeys)))
\end{verbatim}

\subsection{Storey}
\label{sec:orgheadline10}

A storey has a matrix of \emph{tiles}. 
It's likely that we will change the representation of a story to 
a sparse matrix in the future, so abstracting its representation
is worthwhile prophylaxis (future-proofing).

The coordinate system of any storey has origin \((0, 0)\) so that array indices
and tile coordinates are always identical.

\begin{verbatim}
(defconstant +storey-width+  256)
(defconstant +storey-height+ 256)

(defun make-storey (&key (width +storey-width+)
                      (height +storey-height+))
  (make-array `(,height ,width)
              :initial-element nil))

(let ((m (make-storey :width 10 :height 6)))
  (setf (aref m 2 1) :hi)
  m)
\end{verbatim}

\subsection{Tile}
\label{sec:orgheadline11}

Each tile contains exactly one (possible nil)
\emph{geo}, zero or one \emph{critters}, and a bag of \emph{treasures}. We allow nil geos
because most tiles will have nothing interesting in them. A nil tile means the
same as a nil geo in a non-nil tile.

\begin{verbatim}
(defclass tile ()
  ((geo       :accessor tile-geo       :initform nil :initarg :geo)
   (critter   :accessor tile-critter   :initform nil :initarg :critter)
   (treasures :accessor tile-treasures :initform nil :initarg :treasures)
   (lighted   :accessor tile-lighted   :initform nil :initarg :lighted
  )))

  (let ((m (make-storey :width 10 :height 6)))
    (setf (aref m 2 1)
          (make-instance 'tile))
    m)
\end{verbatim}

\subsection{Room}
\label{sec:orgheadline12}

A \emph{room} is a rectangular region of tiles in a storey. It has a reference to its
storey and a reference to a box specifying the top and left coordinates of the
room with respect to the storey's origin \((0,0)\), and a width and height.

\subsection{Geo}
\label{sec:orgheadline46}

A geo could be nil, meaning ``nothing interesting here,'' or a wall, door, rock, or trap.

\begin{verbatim}
(defclass geo ()
  ((kind :accessor geo-kind :initform nil :initarg :kind)))
\end{verbatim}

\subsubsection{Wall}
\label{sec:orgheadline14}
\begin{enumerate}
\item Inscribed
\label{sec:orgheadline13}
\end{enumerate}
\subsubsection{Door}
\label{sec:orgheadline20}
\begin{enumerate}
\item Open
\label{sec:orgheadline15}
\item Closed
\label{sec:orgheadline16}
\item Locked
\label{sec:orgheadline17}
\item Spiked
\label{sec:orgheadline18}
\item Broken
\label{sec:orgheadline19}
\end{enumerate}
\subsubsection{Rock}
\label{sec:orgheadline25}
\begin{enumerate}
\item Granite
\label{sec:orgheadline21}
\item Quartz
\label{sec:orgheadline22}
\item Magma
\label{sec:orgheadline23}
\item Lava
\label{sec:orgheadline24}
\end{enumerate}
\subsubsection{Trap}
\label{sec:orgheadline45}
\begin{enumerate}
\item Gas
\label{sec:orgheadline30}
\begin{enumerate}
\item Poison
\label{sec:orgheadline26}
\item Drug
\label{sec:orgheadline27}
\item Blindness
\label{sec:orgheadline28}
\item Fear
\label{sec:orgheadline29}
\end{enumerate}
\item DimMak
\label{sec:orgheadline31}
\item Dart
\label{sec:orgheadline34}
\begin{enumerate}
\item Poison
\label{sec:orgheadline32}
\item Drug
\label{sec:orgheadline33}
\end{enumerate}
\item Fire
\label{sec:orgheadline35}
\item Boulder
\label{sec:orgheadline36}
\item Flood
\label{sec:orgheadline37}
\item Curse
\label{sec:orgheadline38}
\item Ice
\label{sec:orgheadline39}
\item Immoblization
\label{sec:orgheadline40}
\item Lightning
\label{sec:orgheadline41}
\item Pit
\label{sec:orgheadline42}
\item Hole
\label{sec:orgheadline43}
\item Teleport
\label{sec:orgheadline44}
\end{enumerate}
\subsection{Critter}
\label{sec:orgheadline49}
\subsubsection{Me}
\label{sec:orgheadline47}
\subsubsection{Monster}
\label{sec:orgheadline48}
\subsection{Treasure}
\label{sec:orgheadline54}
\subsubsection{Potion}
\label{sec:orgheadline50}
\subsubsection{Scroll}
\label{sec:orgheadline51}
\subsubsection{Armor}
\label{sec:orgheadline52}
\subsubsection{Weapon}
\label{sec:orgheadline53}

\section{Rendering}
\label{sec:orgheadline59}

\subsection{Screen}
\label{sec:orgheadline56}
\subsection{Window}
\label{sec:orgheadline57}
\subsection{Scroll-state}
\label{sec:orgheadline58}

\section{Me}
\label{sec:orgheadline80}

\subsection{Attributes}
\label{sec:orgheadline77}
\subsubsection{Dynamic}
\label{sec:orgheadline65}
\begin{enumerate}
\item HitPoints
\label{sec:orgheadline60}
\item Mana
\label{sec:orgheadline61}
\item Energy
\label{sec:orgheadline62}
\item Rage
\label{sec:orgheadline63}
\item Focus
\label{sec:orgheadline64}
\end{enumerate}
\subsubsection{Static}
\label{sec:orgheadline76}
\begin{enumerate}
\item Strength
\label{sec:orgheadline66}
\item Wisdom
\label{sec:orgheadline67}
\item Constitution
\label{sec:orgheadline68}
\item Stamina
\label{sec:orgheadline69}
\item Intellect
\label{sec:orgheadline70}
\item Charisma
\label{sec:orgheadline71}
\item Agility
\label{sec:orgheadline72}
\item Dexterity
\label{sec:orgheadline73}
\item Versatility
\label{sec:orgheadline74}
\item Mastery
\label{sec:orgheadline75}
\end{enumerate}
\subsection{Classes}
\label{sec:orgheadline78}
\subsection{Races}
\label{sec:orgheadline79}
\section{Treasures}
\label{sec:orgheadline110}
\subsection{Armor}
\label{sec:orgheadline92}
\subsubsection{Head}
\label{sec:orgheadline81}
\subsubsection{Shoulders}
\label{sec:orgheadline82}
\subsubsection{Chest}
\label{sec:orgheadline83}
\subsubsection{Arms}
\label{sec:orgheadline84}
\subsubsection{Wrists}
\label{sec:orgheadline85}
\subsubsection{Hands}
\label{sec:orgheadline86}
\subsubsection{Pants}
\label{sec:orgheadline87}
\subsubsection{Feet}
\label{sec:orgheadline88}
\subsubsection{Neck}
\label{sec:orgheadline89}
\subsubsection{Trinkets}
\label{sec:orgheadline90}
\subsubsection{Rings}
\label{sec:orgheadline91}
\subsection{Weapons}
\label{sec:orgheadline108}
\subsubsection{One-Handers}
\label{sec:orgheadline98}
\begin{enumerate}
\item Swords
\label{sec:orgheadline93}
\item Daggers
\label{sec:orgheadline94}
\item Maces
\label{sec:orgheadline95}
\item Clubs
\label{sec:orgheadline96}
\item Fists
\label{sec:orgheadline97}
\end{enumerate}
\subsubsection{Two-Handers}
\label{sec:orgheadline103}
\begin{enumerate}
\item Staves
\label{sec:orgheadline99}
\item Swords
\label{sec:orgheadline100}
\item Maces
\label{sec:orgheadline101}
\item Clubs
\label{sec:orgheadline102}
\end{enumerate}
\subsubsection{Ranged}
\label{sec:orgheadline107}
\begin{enumerate}
\item Guns
\label{sec:orgheadline104}
\item Bows
\label{sec:orgheadline105}
\item Crossbows
\label{sec:orgheadline106}
\end{enumerate}
\subsection{}
\label{sec:orgheadline109}
\section{Players}
\label{sec:orgheadline113}
\subsection{AI}
\label{sec:orgheadline111}
\subsection{Bots}
\label{sec:orgheadline112}
Emacs 24.5.1 (Org mode 8.2.10)
\end{document}